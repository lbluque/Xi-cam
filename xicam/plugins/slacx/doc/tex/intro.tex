\section{Introduction}
\label{sec:introduction}

The \verb|slacx| software package provides 
a fast and lean platform for processing image-like data.
It is being developed to perform analysis of x-ray diffraction patterns 
for ongoing projects at SLAC/SSRL.
At the core of \verb|slacx| is a workflow engine
that allows users to either design and execute a single workflow,
run a known workflow on a batch of inputs,
or monitor a filesystem and execute a workflow on arriving files in real time.
Other features include:
\begin{itemize}
\item A plugin module for using slacx and its components 
    in the \verb|xi-cam| software package
\item An operation development interface 
    for users with little or no programming experience 
    to write their own processing routines 
\item Networking modules for directing remote (distributed) computations 
    and communicating with remote filesystems to store and retrieve data
\end{itemize}
Some long-term goals of \verb|slacx| are: 
\begin{itemize}
\item To eliminate the development of redundant analysis routines,
    reducing the potential for bugs 
    and increasing overall refinement and efficiency of the workflow
\item To streamline analysis and standardize storage 
    so that processed data is easily accessible for future reference
\item To interface with experimental equipment
    to enable model-driven feedback 
    targeting specific engineering objectives
\end{itemize}

The \verb|slacx| developers would love to hear from you
if you have wisdom, thoughts, haikus, bugs, artwork, suggestions, or limericks.
Get in touch with us at \verb|slacx-developers@slac.stanford.edu|.




